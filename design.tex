\section{Design Principles}
To learn about current practices and unsupported needs in presentation technology, we conducted multiple rounds of interviews with classroom lecturers, online lecturers, graduate student TAs, and undergraduate students, with varied experiences in giving and listening to presentations. 

The interviews led us to the following key insights which informed the design of our interface.

\textbf{Flexibility in presentations is preferred for interactive or informal settings.} For settings such as research conferences or business meetings, where there is a tight time constraint and little room for presenter-audience interaction, people preferred scripted presentations with electronic slides. However, for settings such as lectures, tutoring sessions, or brainstorming meetings, people preferred to have some flexibility and often used inking as part of their presentations. Common strategies included using a blackboard or projecting a slide or transparency on the board and inking on top of them. Several lecturers purposefully left out blank spaces on their slides to fill in by inking during the lecture. Inking helped presenters to draw the audience's attention by  writing in real-time or making annotations, and to adjust the content or order of the presentation. 

\textbf{Presenters want flexibility over prepared contents than complete improvisation.} Even for informal settings, presenters had the bulk of the content planned and prepared beforehand, for example, in the form of lecture notes, worksheets or slides. The type of flexibility that presenters wanted was the ability to make small-scale adjustments on-the-fly, for instance, omitting a part of the content, adding minor changes such as a line of text or annotations, and determining the order of the contents. Depending on the subject matter, inking in real time allowed presenters to have fine-grained control over the pace the presentation. \val{Especially, When it's describing a process or math formula pace is important.} 

\textbf{Presenters and audience care about beauty.} People complain about messy handwriting. Presenters are not satisifed, and audience cannot read them afterwards. It's difficult to do it well while talking at the same time. It takes too much time to draw. In case of online lecture, record voice and visual separately, and edit visual many times. Prepare diagrams beforehand. Difficulty of choosing pen color etc.

From these we extract key design goals: 

1. Minimize cognitive load of the presenter. Should be modeless.
In previous study of inking during presentation [cite], presenters use
features (e.g., color, highlighting) parsimoniously.

2. Increase flexibility. fine-grained reveal. Also reduces prep time.

3. Retain direct control. Instead of automatic beautification / layout management. 

\wil{Before getting into the details of the method, I think we should
  describe the key aspects of our design (fine-grained reveal and
  space manipulation, right?) and connect them to our design principles.}

