\section{Limitations and Discussion}
\val{In general, I had a hard time trying to group/order the contents in this section. The second part about interactivity seems less related to the other two. I almost feel like it's less relevant and that we should focus on 1 and 3 (No specific reason why it's ordered this way. }
Here, we consider some limitations of our design, and discuss several observations and lessons learned about designing future presentation interfaces.\\

\textbf{Different Ways to Reveal.}  Several presenter participants expressed that they would like an even quicker way than scribbling to reveal. Indeed, scribbling over a large area to reveal an image or a long text can be tedious, especially if the presenter is not concerned about simulating the real time hand-drawn effect. Supporting different types of revealing mechanisms (e.g., clicking or lasso selection) and allowing instantaneous reveal of certain elements could improve the presentation experience. However, there are several trade-offs to consider.\\
One important consideration is to implement a design that does not increase the burden during preparation or limit the flexibility during presentation. For example, we considered allowing the presenters to specify at setup time elements that will be revealed instantaneously upon clicking. But, this can make preparation tedious akin to grouping elements and setting up animations on PowerPoint. Moreover, it does not take into account the possibility that presenters can change their mind during preparation about how to reveal.  \\
Another trade-off is between the presenter convenience and the quality of the presentation. Allowing presenters to choose between different revealing gestures has the danger that they may be tempted to always opt for the minimum-effort gesture even at the cost of presentation quality. In fact, in an earlier prototype of our system, we allowed revealing either by lasso selection or slow tracing, and observed that presenters almost always opted for the lasso tool even for \textit{Derivation} type of content. We felt that this compromised the presentation quality for the audience and instead opted for a design that \textit{forces} the presenter to go at a slower pace.\\
%Anothere possibility is to implement a context aware revealing mechanism that analyzes the slide content and the presenters gesture to decide the best way to reveal (e.g., if the presenter draws a closed curve around an image, the entire image is revealed instantly). However, it is impossible to anticipate all the different use cases (e.g., the presenter may actually want to reveal a closed contour inside an image), and erroneous predictions can be harmful. \val{Maybe I should emphasize that "best way" is ambiguous, rather than the fact that the system might misinterpret the user's intent (which is how it reads now).}\\

Similarly, some presenters as well as audience participants complained that showing the scribbles before revealing the foreground elements is aesthetically less pleasing or even distracting. However, other audience members commented that the scribbles served as a helpful cue to draw their attention to the location where the information was going to appear. The short transition time between when the scribbles appear and when the underlying elements are revealed can also give audiences time to turn their focus \val{I have not measured this time but I would say about 0.5sec. Should we mention this?} . In general, finding the sweet spot between real time drawing and instantaneous display that is convenient for the presenter and appropriately paced for the audience is a non-trivial problem.

\textbf{Creating Flexible Interactivity.} We focused mainly on presenting slides with static visuals, and provide only the most basic interaction with the displayed contents, namely, to shift them in order to create empty space. In our preliminary formative interview, several participants expressed the desire to present interactive content that could be controlled on-the-fly, for instance, to explain steps of a computer algorithm or a physics diagram. Previous approaches in this area rely on domain specific knowledge \cite{laviola2007mathpad,alvarado2001preserving} or scripted animations \cite{zongker2003creating}.  Exploring the range of interactivity relevant to presentations, and designing interactions that align with our goals (maximizing flexibility without minimizing presenter burden) is an important area for future work. \val{I feel like Rubaiat Kazi's work on Kitty/Draco is also relevant here, but I am not sure how to bring them in.}

\textbf{Balancing Presenter Convenience vs. Audience Engagement.} Presenter interfaces are inherently two-sided. On the one hand, it is easy to focus on the presenters since they are the direct users of the interface. However, it is equally, if not more, important to take into account the audience's point of view since they are the intended consumers of the output of the interface. As previously noted, presenters and audience may have conflicting interests. Presenters may prefer the easiest path, which may not be most informative path for the audience. Moreover, these preferences are context dependent. \val{point to our formative work?} Balancing the two sides can be tricky but also opens up opportunity. Works such as \cite{seth2010powerpoint, cross2013typerighting} that systematically compare the effect of different presentation styles can help. Also once we define effective presentation styles, designing interfaces that assist and \textit{nudge} the presenters to implement such styles is important. 
