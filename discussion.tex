\section{Limitations and Discussion}
%\val{In general, I had a hard time trying to group/order the contents in this section. The second part about interactivity seems less related to the other two. I almost feel like it's less relevant and that we should focus on 1 and 3 (No specific reason why it's ordered this way. }
%\wil{I moved topic 2 to the end. I agree that it is less related than the other two but perhaps still nice to include.}
Here, we consider some limitations of our design, and discuss several observations and lessons learned about designing future presentation interfaces.

\textbf{Different Ways to Reveal.}  Several presenters in the first user study wanted an even quicker way than scribbling to reveal content. Indeed, scribbling over large areas to reveal images or long lines of text can be tedious, especially if the presenter is not concerned about simulating the real-time hand-drawn effect. 
%
In addition, some participants across both user studies complained that showing the scribbles before revealing the foreground elements is less aesthetically pleasing or even distracting. Conversely, other audience members commented that the scribbles served as helpful cues to draw attention to where information was about to appear. %The short transition time between when the scribbles appear and when the underlying elements are revealed can also give audiences time to turn their focus \val{I have not measured this time but I would say about 0.5sec. Should we mention this?}. \wil{I'm actually not clear about this point. Doesn't the transition time depend on how much scribbling is required? And what does it mean for audiences to turn their focus? Can we skip this last sentence?}
%
In general, supporting different types of revealing mechanisms (e.g., clicking or lasso selection) and allowing instantaneous reveal of certain elements could improve the presentation experience. However, such a design should not increase the preparation effort or limit flexibility during presentation.
%
For example, we considered allowing presenters to specify at setup time elements that can be revealed instantaneously upon clicking or tapping. However, this can make preparation tedious, akin to grouping elements and setting up animations in PowerPoint. Moreover, it does not allow presenters to change their mind during preparation about how to reveal elements. 

\textbf{Presenter Convenience vs. Presentation Quality.} The question of what revealing mechanisms to support points to a more fundamental tension with presenter interfaces. 
%
On the one hand, making the workflow more convenient for presenters is an important objective. However, it is at least as important to take into account the audience's point of view, since they are the intended consumers of the presentations themselves.
%
For example, providing a larger set of revealing gestures may tempt presenters to always opt for the minimum-effort gesture at the cost of presentation quality. In fact, in an earlier prototype of our system, we allowed revealing either by lasso selection or slow tracing, and observed that presenters almost always opted for the lasso tool even for the \textit{Derivation} type of content, which compromised presentation quality for the audience. Instead, we opted for a design that \textit{forces} the presenter to go at a slower pace.
%
More broadly, designing presentation tools that achieve the right balance between the needs of presenters and audience members remains an interesting direction for future work. In this vein, previous research that systematically compares the effect of different presentation styles (e.g.,~\cite{seth2010powerpoint, cross2013typerighting}) may provide valuable guidance for new interfaces that assist and \textit{nudge} presenters towards the most effective communication techniques.

\textbf{Creating Flexible Interactivity.} Our work focuses on presenting slides with static visuals, and \interface\ supports limited interaction with the displayed content (i.e., shifting elements to create empty space). In our formative interviews, several participants expressed the desire to present interactive content that can be controlled on-the-fly, for instance, to explain the steps of a computer algorithm or a physics diagram. 
%
While there are many existing approaches to creating interactive diagrams, designing authoring and presentation tools that are both convenient and flexible is a challenging problem that bears further exploration.

%\wil{I left out the references below. Not sure if we need them.\\
%\\
%Some previous approaches to creating interactive diagrams rely on domain specific knowledge \cite{laviola2007mathpad,alvarado2001preserving}. More general-purpose techniques create parameterized graphics and animations via programming~\cite{zongker2003creating} or visual authoring of simple state machines~\cite{kazi2014kitty}.
%}


\if 0
Presenter interfaces are inherently two-sided. On the one hand, it is easy to focus on the presenters since they are the direct users of the interface. However, it is equally, if not more, important to take into account the audience's point of view since they are the intended consumers of the output of the interface. As previously noted, presenters and audience may have conflicting interests. Presenters may prefer the easiest path, which may not be most informative path for the audience. Moreover, these preferences are context dependent. \val{point to our formative work?} Balancing the two sides can be tricky but also opens up opportunity. Works such as \cite{seth2010powerpoint, cross2013typerighting} that systematically compare the effect of different presentation styles can help. Also once we define effective presentation styles, designing interfaces that assist and \textit{nudge} the presenters to implement such styles is important. 

In general, finding the right balance between real-time drawing and instantaneous display that is both convenient for the presenter and appropriately paced for the audience is an interesting problem that bears further exploration.

Supporting different types of revealing mechanisms (e.g., clicking or lasso selection) and allowing instantaneous reveal of certain elements could improve the presentation experience. However, there are several trade-offs to consider.
%
One important consideration is to implement a design that does not increase the preparation effort or limit flexibility during presentation. For example, we considered allowing presenters to specify at setup time elements that can be revealed instantaneously upon clicking or tapping. However, this can make preparation tedious, akin to grouping elements and setting up animations in PowerPoint. Moreover, it does not allow presenters to change their mind during preparation about how to reveal elements. 
%
Another trade-off is between presenter convenience and the quality of the presentation. Providing a larger set of revealing gestures may tempt presenters to always opt for the minimum-effort gesture at the cost of presentation quality. In fact, in an earlier prototype of our system, we allowed revealing either by lasso selection or slow tracing, and observed that presenters almost always opted for the lasso tool even for the \textit{Derivation} type of content, which compromised presentation quality for the audience. Instead, we opted for a design that \textit{forces} the presenter to go at a slower pace.
%Anothere possibility is to implement a context aware revealing mechanism that analyzes the slide content and the presenters gesture to decide the best way to reveal (e.g., if the presenter draws a closed curve around an image, the entire image is revealed instantly). However, it is impossible to anticipate all the different use cases (e.g., the presenter may actually want to reveal a closed contour inside an image), and erroneous predictions can be harmful. \val{Maybe I should emphasize that "best way" is ambiguous, rather than the fact that the system might misinterpret the user's intent (which is how it reads now).}\\
\fi


