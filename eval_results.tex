\subsection{Discussion of Study Results}
Overall, presenters found \interface easy to use for setting up and delivering presentations. They were satisfied with the quality of the presentations recorded using our interface, and emphasized that the appearance of real time inking had an engaging effect. The preference of presentation tool depended on the content type. Both the presenters and the audiences preferred \interface for text-centered or process-driven content. Below we discuss each of these aspects in more detail. 

\definecolor{color1}{rgb}{0.92, 0.83, 0.78}
\definecolor{color2}{rgb}{0.91, 0.66, 0.66}
\definecolor{color3}{rgb}{0.68, 0.68, 0.96}
\begin{figure}
\centering
\begin{tikzpicture}
\begin{axis}[
    ybar,
    enlargelimits=0.15,
    enlarge y limits={upper=0},
    legend style={at={(0.5,-0.15)},
      anchor=north,legend columns=-1},
    symbolic x coords={Easy Setup, Easy Delivery, Satisfaction},
    xtick=data,
%    nodes near coords,
%    nodes near coords align={vertical},
    scaled y ticks = false,
    ymin =0,
    ylabel={Score},
    ]
\addplot[style={fill=color1}, error bars/.cd, y dir=both, y explicit]
coordinates {
(Easy Setup, 3.6) += (0,1.0) -= (0,1.0)
(Easy Delivery, 4.2) += (0,0.7) -= (0,0.7)
(Satisfaction, 3.3)+= (0,0.8) -= (0,0.8)};
\addplot[style={fill=color2}, error bars/.cd, y dir=both, y explicit] 
coordinates {(Easy Setup, 3.9) += (0,0.2) -= (0,0.1)
(Easy Delivery, 3.3) += (0,0.2) -= (0,0.1)
(Satisfaction, 2.8)+= (0,0.2) -= (0,0.1)};
\addplot[style={fill=color3}, error bars/.cd, y dir=both, y explicit] 
coordinates {(Easy Setup, 4.6) += (0,0.2) -= (0,0.1)
(Easy Delivery, 3.8) += (0,0.2) -= (0,0.1)
(Satisfaction, 3.7)+= (0,0.2) -= (0,0.1)};
\legend{BaselinePPT, BaselineInk, \interface (Ours)}
\end{axis}
\end{tikzpicture}
\caption{Chart 1}
\label{eval_chart1}
\end{figure}

\textbf{\interface\ makes it easier to prepare slides.}\\
\val{I feel like we should be more precise about the word preparation. Preparation can include studying the material, authoring the slide (plus setting up the slide--with animation effects etc), and rehearsing the slides. Here, I am trying to focus on the "set up part."}\\
Presenters found it easiest to set up the slides using our interface, followed by BaselineInk and then BaselinePPT (Figure~\ref{eval_chart1}). With \interface, most presenters did not do any extra work (revealing or writing beforehand) to set up the slides, but used them as is. In the few cases, where they pre-revealed parts of the foreground, they expressed that the required effort was minimal. \\
In comparison, although our participants were familiar users of PowerPoint, they found the effort to setup animation effects tedious. To quote \textit{U6}, "\textit{It was cumbersome to add animations to each individual object and get the timing right... sometimes I decided I wanted to add an animation, but then had to figure out where to insert it in the existing animations sequence.}"\\
In the BaselineInk condition, similar to in \interface, most participants used the slide as is, without writing contents ahead of time. However, the reasons were different. In \interface, presenters could rely on the pre-authored foreground to effortlessly reproduce those contents (by revealing) to the audience in real time during delivery. In the BaselineInk condition, presenters had to manually draw the foreground contents. They could do this ahead of time, which would mean losing the real time effect, or they could do it during delivery. Either way, the effort was more "\textit{daunting (U12)}" and  "\textit{time consuming (U8)}", and the result was aesthetically less satisfactory (\textit{U1, U5, U8}).  
\val{Kruskall-Walis: chi-squared; 4.865, p = 0.0878}

\textbf{For presentation delivery, \interface involves comparable effort to PowerPoint.}\\
\val{Kruskall-Walis chi squared: 4.044, p = 0.13}
Presenters found it easiest to deliver the presentation using BaselinePPT, followed by our interface, and then BaselineInk. The difference between BaselinePPT and our interface (Mann-Whitney U test: $Z = 1.18, p = .24$) was less significant than between that of BaselinePPT and BaselineInk ($Z = -1.88, p = .06$). \\
It is not surprising that BaselinePPT required minimum effort. The only interaction presenters used during delivery was to press a single button to advance the slide animations. That being said, when the animation involved more than a few steps, it was common for presenters to make a mistake. 3 out of 12 participants forgot to advance the animation at the right time at least once, and only realized it at the next animation step. They had to either repeat the verbal explanation or quickly skip through the missed animations. In addition, 2 participants reported that they forgot to set up a desired animation step and only realized it during delivery. \\
Presenters found inking in \interface straightforward, but they noted that inking to reveal was extra effort compared to pushing a single button. Presenters seemed to prefer the path of minimum effort. For example, with the exception of the math derivation content, they mostly used a fast scribbling or strike-through gesture to reveal. As several presenters mentioned, this had the downside that the audience would initially see the scribbled ink strokes before the underlying pixels were revealed, which could be potentially distracting and aesthetically less pleasing. Some users suggested that they would prefer an even faster gesture such as clicking or circling to reveal large parts. That being said, they also expressed the idea that, for the audience it could be better  $"to have the word appear after the inking instead of just after clicking like in powerpoint." (U1)$  \val{I want to discuss the downsides of this path (e.g., less engaging for the audience, need setup (grouping of elements) but I don't want digress too much here.  What's a good phrase to signal the readers that we'll discuss this issue further later?} On the other hand, users appreciated the automatic color selection for annotation and the modeless switching between revealing and inking. \val{Should we mention here or in the Protocol, that space manipulation was not introduced the the participants for this study? Prefer in the protocol.}\\
As expected, BaselineInk was the . 


2. As easy to present
3. Presenters were satisfied with the result. Audience satisfaction depended on subject. 
\val{chi-squared = 3.83 / p = 0.1473}
4. Content-tool affects each other.
