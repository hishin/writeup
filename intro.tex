\section{Introduction}

Presentations are an important component of both classroom and online instruction. 
%
They allow presenters to communicate concepts by combining visual content with spoken explanations.
%
As a result, tools for authoring and delivering presentations have a significant influence on how people teach and learn.
%
Today, the two dominant types of presentation technology are slides and blackboards.

%Among other roles, presentation media allow persistent display of information and serve as a shared view between the presenter and the audience \cite{anderson2004beyond}. 

Slides allow presenters to refine the appearance and organization of material in advance. As such, they are most convenient for information-rich content like images or detailed diagrams and charts. 
%
However, pre-authored slides restrict how content can be revealed during the presentation. Rather than displaying all of the slide content at once, which can make it difficult for the audience to know where to focus, presenters often set up animation effects to reveal visual elements incrementally. Since the sequence and granularity of these animations are determined ahead of time, it is difficult to add new content or change the order of reveals during the presentation, for instance, in response to the audience.
%
Animations also display information quickly and hasten the pace of the presentation. Slide lectures tend to show more information in a shorter period of time than blackboard lectures \cite{lanir2008observing}. \wil{Is this last point a positive or negative point? Not clear from the wording right now.}

As an alternative, some presenters prefer to write on physical blackboards or ink in real-time on virtual displays. This presentation style is direct and intuitive, and in contrast to pre-authored slides gives presenters full control over how content is displayed.
%
On the other hand, writing and talking at the same time is cognitively demanding. As a result, bad handwriting and poor layouts (e.g., running out of room while writing) are common in blackboard-style presentations. It is also difficult to incorporate complex visuals since they must be created from scratch during the presentation. This often results in long pauses or rambling, repetitive explanations as presenters focus on drawing or writing.

Some presenters blend the two modes of presentation, for example, by projecting slides onto a board and inking on top of them. Recently, PowerPoint and Keynote, also allow presenters to ink over electronic slides in presentation mode. However, in these approaches, the ink and underlying slides are treated as completely separate layers of content that retain their individual drawbacks. 
%
The slide content remains fixed and inflexible while real-time inking still requires the user to draw carefully (albeit with the help of the slide content as a reference).

To address these limitations, we propose \interface \footnote{A charm for revealing invisible ink\cite{rowling1997harry}}, a presentation interface that combines the advantages of slides and inking.
%
In \interface, presenters use pre-authored slides to prepare the presentation content. However, instead of specifying animations beforehand, presenters use inking interactions to flexibly display the content on demand. By inking, presenters can control the pace at which pre-authored material is revealed, as if writing in real-time. At the same time, they do not have to worry about writing neatly or carefully arranging all the visual elements on the screen. Our system also allows improvised annotations and enables presenters to create blank space inside the slide to insert new content during the presentation. \interface\ supports all of these functions as modeless interactions.
% contents to be inked at presentation time. During the preparation phase, presenters can take time to write neatly, plan the layout carefully, and also include typed text or images as they would for a slide presentation. Unlike slide presentations, presenters need not determine beforehand the precise order or granularity of how these contents will be displayed (i.e., animation order and groups). Instead, presenters simply decide which parts of the content is initially hidden from the audience. During presentation, the hidden contents are visible to the presenters and serve as a guide. Inking either reveals the pre-prepared contents to the audience or appears as an impromptu annotation depending on the context. Presenters can adjust the content, its order, and pace on demand by deciding whether, when, and how to reveal each part. We also provide an interaction to modify the layout on-the-fly by creating or reducing blank space, similar to \cite{yoon2013texttearing}. In order to minimize the cognitive load during presentation, all of these interactions are provided as modeless pen interactions. \\

We evaluate our interface from the perspective of presenters and the audience and compare \interface\ against two baselines, representing conventional slide and inking tools. Presenters found \interface\ easy to use for preparing and delivering presentations. They were also generally pleased with the quality of the presentation produced using our interface. 
%
Overall preferences between presentation interfaces depended on the type of content.
%
Both presenters and audience members clearly preferred \interface\ for text-heavy, process-driven material (e.g., explaining algorithms or mathematical derivations), and rated our system as comparable to the baseline slide interface for presenting complex, multi-part diagrams. 
%
For simple bullet point slides, participants preferred the baseline slide presentation tool. \wil{Not sure if this sounds a bit weak. We could cut this last sentence and just focus on the positives.}

To summarize, our work makes several contributions: 
\begin{itemize}
  \item The design and architecture of a new presentation interface, \interface, that combines inking with pre-authored slides.
\item A set of modeless inking interactions that analyze the underlying slide content to help presenters reveal, annotate, and create extra space during a presentation.
  \item An evaluation from the perspective of presenters and the audience that compares \interface\ against two baseline interfaces.
%  \item Based on formative interviews and observations from our user study, we discuss design directions for future presentation interfaces. 
\end{itemize}



% \textbf{We propose inking as a main modality to present slide contents in a continuous, flexible and direct manner}. By inking over to reveal content, users regain flexibility, continuity and direct control over the presentation style, without losing the elegance of the prepared content.  
%Slides are fit for scripted presentations.

%There are previous work to blend the two modes. [Classroom Presenter] or recent versions of [PowerPoint] allow you to ink on slides. But these tools treat the two modes as separate. Ink and slide remain separate layers retaining their characteristics. The slide layer remains rigid and fixed; and ink is placed on top of it.\\

%35mm slide projectors first came into widespread use during the 1950s, but recently the popularization of slide authoring tools like PowerPoint, Keynote or Google Slides made electronic slides became much more common and easy. Despite its advantages--e.g., easy to share, archive, include multimedia--slides also have critical drawbacks. 
%(1) Slide presentations are \textbf{rigid}: All of the editing and preparation is done ahead of time and fixed at the time of the presentation. There is no flexibility to change the order or content of the presentation during performance. 
%(2) Slide presentations are \textbf{discrete}: Information is divided and presented in chunks. First the entire content is divided into separate slides, and within each slide text or graphics are presented in chunks, usually using animation effects. The appearance or transition of information is sudden and discrete.
%(3) Slide presentations are \textbf{indirect}: The action of the presenter (i.e., pressing "next" to advance the slide) is removed from the effect on the content. For example, this makes it easy for presenters to forget or skip an animation sequence.\\
%
%Inking on surface is an alternative or supplement. Inking includes blackboard or transparency and overhead projector. Inking complements the characteristics of slides. Inking is flexible, continuous and direct, but this has its own downsides.
%(1) Inking is \textbf{flexible} since all of the presentation is done on the fly. But, this means there is a lot of cognitive overload for the presenter in order to decide the content, layout and order of the presentation on the fly. 
%(2) Inking is \textbf{continuous}: Since the presenter writes or draws in real time, content is presented in a continuous way. This is useful for information-loaded contents or where order within the content is important, for example, derivation of a math formula or describing a temporal process. However, it is difficult to time the presentation or present a lot of information. Inking lectures tend to be slower paced [citation].
%Finally, (3) inking is direct. Users actions (drawing, erasing, underlying) are all directly translated into content. This requires a lot of attention, skill on the part of the presenter. Often, presenters are reluctant to write because of their messy handwriting. Also content is limited to text or simple diagrams.



% Mainly two types of presentation technology used in classroom (and online?) instruction: drawing on surface and creating electronic slides.

% Drawing on surface is engaging (cite), and flexible (improvise). But it can be hard to talk and draw at the same time. Also, hard to alter the pace; can't speed up if you have to draw everything.

% Electronic allow authors to prepare content ahead of time, which makes it less taxing to present (just hit next). But it's also less flexible; can't change the order or granularity of animations. Moreover, if you want very fine-grained reveals, that takes a lot of effort at preparation time.

% Aparecium combines convenience of pre-authored slides with the flexibility and engaging quality of inking. The one key idea is to use inking as a form of fine-grained reveal. Authors prepare content ahead of time, then indicate which parts should be hidden. At presentation time, authors ink, but their strokes are used to reveal corresponding parts of the underlying content. The other key idea is to allow for convenient space manipulation. This approach has several benefits:
% + Authors can prepare as much or as little as they want
% + They have flexibility at presentation time to reveal at different granularities and orders
% + Inked content is always neatly written and arranged (at least as well as the prepared version)
% + No modes: revealing and adding new annotations are all done with the same inking interaction, and user does not have to switch pen colors or styles

% Main contributions are in reveal algorithm, space-pushing technique. We evaluate presentation experience and quality of results compared to baselines.  