\section{Introduction}

Presentation technology has a significant influence on how we learn and teach. Among other roles, presentation media allow persistent display of information and serve as a shared view between the presenter and the audience \cite{anderson2004beyond}. \\

Today, there are mainly two types of presentation technology used in classroom instruction: chalkboards, or more generally, writing or drawing on surface, and electronic slides. Writing or drawing on surface is a direct and intuitive way of communication, but doing it well in the context of a presentation is challenging. For example, writing and talking at the same time can be cognitively demanding. Bad handwriting and poor layouts are common problems expressed by both the presenters and the audience. It is also difficult to incorporate complex diagrams or images. Electronic slides, on the other hand, provide a means of showing information-rich content with improved organization by advanced preparation of material. However, pre-prepared electronic slides come at the expense of the ability to freely adjust the presentation during performance, for example, in response to audience reaction. \val{maybe talk about pace}\\

Recently presentation softwares, such as PowerPoint \cite{powerpoint2017} and Keynote \cite{keynote2017}, also allow inking during presentation. However, these tools treat ink and slide as separate layers, each retaining its own drawbacks. Inking well on-the-fly is still challenging, and slide contents remain unadjustable. \\

To address these problems, we propose \interface\val{Add footnote}, a presentation interface that seamlessly blends inking with slides. In \interface, presenters can pre-prepare contents to be inked at presentation time. During the preparation phase, presenters can take time to write neatly, plan the layout carefully, and also include typed text or images as they would for a slide presentation. Unlike slide presentations, presenters need not determine beforehand the precise order or granularity of how these contents will be displayed (i.e., animation order and groups). Instead, presenters simply decide which parts of the content is initially hidden from the audience. During presentation, the hidden contents are visible to the presenters and serve as a guide. Inking either reveals the pre-prepared contents to the audience or appears as an impromptu annotation depending on the context. Presenters can adjust the content, its order, and pace on demand by deciding whether, when, and how to reveal each part. We also provide an interaction to modify the layout on-the-fly by creating or reducing blank space, similar to \cite{yoon2013texttearing}. In order to minimize the cognitive load during presentation, all of these interactions are provided as modeless pen interactions. \\

\val{Summary of evaluation}\\
\val{Summary of contribution}\\

% \textbf{We propose inking as a main modality to present slide contents in a continuous, flexible and direct manner}. By inking over to reveal content, users regain flexibility, continuity and direct control over the presentation style, without losing the elegance of the prepared content.  
%Slides are fit for scripted presentations.

%There are previous work to blend the two modes. [Classroom Presenter] or recent versions of [PowerPoint] allow you to ink on slides. But these tools treat the two modes as separate. Ink and slide remain separate layers retaining their characteristics. The slide layer remains rigid and fixed; and ink is placed on top of it.\\

%35mm slide projectors first came into widespread use during the 1950s, but recently the popularization of slide authoring tools like PowerPoint, Keynote or Google Slides made electronic slides became much more common and easy. Despite its advantages--e.g., easy to share, archive, include multimedia--slides also have critical drawbacks. 
%(1) Slide presentations are \textbf{rigid}: All of the editing and preparation is done ahead of time and fixed at the time of the presentation. There is no flexibility to change the order or content of the presentation during performance. 
%(2) Slide presentations are \textbf{discrete}: Information is divided and presented in chunks. First the entire content is divided into separate slides, and within each slide text or graphics are presented in chunks, usually using animation effects. The appearance or transition of information is sudden and discrete.
%(3) Slide presentations are \textbf{indirect}: The action of the presenter (i.e., pressing "next" to advance the slide) is removed from the effect on the content. For example, this makes it easy for presenters to forget or skip an animation sequence.\\
%
%Inking on surface is an alternative or supplement. Inking includes blackboard or transparency and overhead projector. Inking complements the characteristics of slides. Inking is flexible, continuous and direct, but this has its own downsides.
%(1) Inking is \textbf{flexible} since all of the presentation is done on the fly. But, this means there is a lot of cognitive overload for the presenter in order to decide the content, layout and order of the presentation on the fly. 
%(2) Inking is \textbf{continuous}: Since the presenter writes or draws in real time, content is presented in a continuous way. This is useful for information-loaded contents or where order within the content is important, for example, derivation of a math formula or describing a temporal process. However, it is difficult to time the presentation or present a lot of information. Inking lectures tend to be slower paced [citation].
%Finally, (3) inking is direct. Users actions (drawing, erasing, underlying) are all directly translated into content. This requires a lot of attention, skill on the part of the presenter. Often, presenters are reluctant to write because of their messy handwriting. Also content is limited to text or simple diagrams.



