\section{Design of \interface}
\wil{Before getting into the details of the method, I think we should
  describe the key aspects of our design (fine-grained reveal and
  space manipulation, right?) and connect them to our design principles.}

\interface uses pre-authored slides, prepared by the presenters ahead of time, in pretty much the same way as they would for a regular electronic presentation. The slides help presenters organize their material and to improve visual aesthetics. However, unlike traditional electronic slides, presenters do not specify beforehand when, how, or in which order the elements in the slide will appear (i.e., animation effects). Instead, they simply specify which elements will be displayed to the audience immediately versus which elements will be revealed in real time. 

\interface's main mode of interaction during delivery is inking. Inking has three functions depending on the context: it can (1) reveal pre-authored slide elements to the audience, (2) add ink strokes on top of the slide, or (3) adjust the slide layout by creating blank space. Inking allows presenters to have flexibility and fine-grained control over when, how much, and how fast to reveal elements on the slide. Presenters can also add extra writings or annotations on top of pre-authored elements, and create blank space if necessary. All of these interactions are implemented as modeless pen interactions. 
 
\subsection{Slide authoring}
\val{Figure showing example of background / foreground / notes}
Slides in \interface can be authored using any existing slide presentation software (e.g., PowerPoint, Keynote, GoogleSlides). They can include typed text or images, as well as, hand drawn ink strokes. Instead of specifying animation effects on these slide elements, presenters separate them into two layers for each slide. The background layer is always visible and it is what the audience sees initially. The foreground layer is initially only visible on the presenter view, but presenters can reveal parts of it to the audience during delivery. ]\val{Presenters also have the option of preparing a third layer, the notes layer, which is only visible on the presenter view and serves as a transparent lecture note placed on top of the slides.} Layers in \interface are represented as bitmap images. 

\subsection{Inking during delivery}
During delivery, presenters ink on top of the pre-authored slides. Inking has three different functions depending on the context. 

\textbf{Reveal}
If the presenter inks over \val{around? I want to somehow convey that inking over does not have to be precise: i.e. on top of the pixels} foreground pixels that have not been revealed to the audience yet, these and neighboring foreground pixels are revealed to the audience in the following way. For each point, $s_i$ in the user's ink stroke, the closest foreground pixel, $p_i$ is computed. If $p_i$ is close enough to $s_i$, a flood-fill is performed starting from $p_i$ to neighboring foreground pixels. The extent of the flood-fill is limited by thresholds on (1) the distance from $p_i$, and (2) the color difference from $p_i$. In order to give presenters finer-grained control over the extent of the reveal, the thresholds vary according to the velocity of the presenter's ink stroke. If the presenter inks slowly, only a small neighborhood close to the original stroke is revealed. This is useful when the presenter wants to simulate writing in real-time, and reveal, for example, a part of a character or a diagram. If the presenter inks quickly (e.g., scribbles), a larger neighborhood is revealed. This allows presenters, for instance, to swiftly reveal an entire image or a line of text without having to ink over them precisely.  \val{Figure showing user tracing over a formula, one character at a time}

\textbf{Annotate}
If the presenter inks over empty pixels on the foreground, or if the foreground pixels around the stroke has already been revealed previously, the ink stroke appears on top of the foreground layer as is. \interface computes the average color of the slide around the stroke and sets the ink color to a complementary color so that the ink stands out from the slide. \val{Figure of annotation over revealed element and annotation on empty background}. 

\textbf{Create Space}
Sometimes, presenters may want extra space to insert new content in the slide, for example, to add an item to an existing list, a word in a sentence, or an extra line of explanation. These situations can arise as a result of a mistake in the preparation phase (e.g., the presenter forgets to list an item), as well as from presenter-audience interaction (e.g., the audience requests extra explanation). \interface allows presenters to create empty space from ink strokes. First, the presenter draws a curve where the empty space is to be created. At the end of the curve, if the presenter holds down the pen for more than 0.5 seconds, the curve turns into a red dashed stroke, indicating that the presenter can start expanding the space around the curve. As the presenter moves the pen along one of the axis-aligned directions, empty space is created and expanded from the curve in that direction.  As the space grows, the foreground content is shifted accordingly. \val{Figure showing the steps.}
