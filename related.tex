\section{Related Work}

\textbf{Presentation Software} PowerPoint \cite{powerpoint2017}, Keynote \cite{keynote2017} and Google Slides \cite{googleslides2017} are among the most popular presentation software in use. These systems make authoring slides very easy through a WYSIWYG graphical interface. Powerpoint and Keynote also offer a variety of animation effects that can be applied to slide elements. However, compared to the flexibility during authoring, interaction during delivery is minimal, and the slideshow is restricted to a predetermined linear path. 

\textbf{Nonlinear Presentations} To address this shortcoming, several research systems looked at supporting nonlinear paths though a presentation. Moscovich et al \cite{moscovich2004customizable} organized slides into nested directed graphs and allowed presenters to choose between multiple paths on the fly. Similarly, Drucker et al. \cite{drucker2006comparing} suggested a method to compare and manage multiple slide presentation paths. Fly \cite{lichtschlag2009fly} and CounterPoint \cite{good2002zoomable} allowed spatial navigation by embedding slides on an infinite canvas and employing zooming user interfaces (ZUIs). Prezi \cite{prezi2017} is a commercial, online platform for authoring zoomable presentations. Whereas these work focus on navigating between slides or the presentation as a whole, the interactions in our work provide flexibility and control within each slide.

\textbf{Controlling Presentations} Another line of work explored alternative techniques to control slide presentations. Palette \cite{nelson1999palette} used physical paper cards to provide random access to slides.  Baudel and Beaudouin-Lafon \cite{baudel1993charade} used hand gestures, and Cheng and Pulo \cite{cheng2003direct} used infrared laser pointer to control presentations. Cao et al. \cite{cao2005evaluation} performed a systematic user study comparing different interaction techniques, including hand gestures, laser pointer and standard mouse/keyboard input. Our work suggests pen interactions \val{inking?} as the main mode to present slides.

\textbf{Inking on Digital Documents} Many systems \cite{yoon2014richreview, marshall1999collaborating, hardock1993marking} support digital inking to make annotations on top of documents. Perhaps most similar in spirit to our work are systems that integrate digital ink with prepared slides. Anderson et al \cite{anderson2007classroom} developed Classroom Presenter, a distributed presentation system, which allowed instructors and students to share digital ink on top of electronic slides. Recently, PowerPoint and Keynote also allow presenters to ink in presentation mode. SMART is another commercial system that supports inking and projected material using an interactive whiteboard \cite{smarttech2017}. While all of these systems combine slides with inking, the underlying slide contents remain inherently separate from the ink on top. In contrast, in our interface, ink interacts with the underlying slide elements to control their appearance in real time. \val{This sentence would also benefit from clarification of inking: as a gesture vs. output.} 

\textbf{Beautifying Ink} To further assist freeform digital inking, researchers have experimented with different methods to beautify the user's freeform ink strokes. Beautification is applied to meet the requirements of specific scenarios, such as geometric diagrams \cite{igarashi1998pegasus, hse2005recognition, fivser2015shipshape}, portraits {xie2014portraitsketch}, handwritings \cite{zitnick2013handwriting} or mathematical diagrams {laviola2007mathpad}. Our work does not modify the user's ink stroke per se, but we achieve a similar effect by making the ink stroke disappear gradually and revealing the underlying pre-authored slide instead.
