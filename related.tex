\section{Related Work}

\textbf{Presentation Software.} The vast majority of presentations today are created with WYSIWYG slide authoring software like PowerPoint \cite{powerpoint2017}, Keynote \cite{keynote2017} and Google Slides \cite{googleslides2017}. 
%
While these tools provide a broad range of content creation features, including the ability to add animation effects to slide elements, 
they offer limited flexibility or control at presentation time. 
%
Presenters can only advance linearly through the predefined sequence of animations and slide transitions.

\textbf{Nonlinear Presentations.} To address this shortcoming, previous research investigates how to support nonlinear paths though a presentation. Moscovich et al \cite{moscovich2004customizable} organize slides into nested directed graphs and allow presenters to choose between multiple paths on the fly. Similarly, Drucker et al. \cite{drucker2006comparing} suggest a method to compare and manage multiple slide presentation paths. Fly \cite{lichtschlag2009fly} and CounterPoint \cite{good2002zoomable} allow spatial navigation by embedding slides on an infinite canvas and employing zooming user interfaces (ZUIs). Prezi \cite{prezi2017} is a commercial, online platform for authoring zoomable presentations. Whereas this work focuses on navigating between slides or the presentation as a whole, the interactions in our work provide flexibility and control within each slide.

\textbf{Controlling Presentations.} Other work explores alternative techniques to control slide presentations. Palette \cite{nelson1999palette} uses physical cards to provide random access to slides, Baudel and Beaudouin-Lafon \cite{baudel1993charade} propose hand gestures, and Cheng and Pulo \cite{cheng2003direct} use an infrared laser pointer to control presentations. Cao et al. \cite{cao2005evaluation} perform a systematic user study comparing different interaction techniques, including hand gestures, laser pointer and standard mouse/keyboard input. Our work suggests pen interactions \val{inking?} as the main mode to present slides.

\textbf{Inking on Digital Documents.} Many systems \cite{yoon2014richreview, marshall1999collaborating, hardock1993marking} support digital inking to make annotations on top of documents. Perhaps most similar in spirit to our work are systems that integrate digital ink with prepared slides. Anderson et al.~\cite{anderson2007classroom} propose Classroom Presenter, a distributed presentation system that allows instructors and students to share digital ink on top of electronic slides. Recently, PowerPoint and Keynote added support for presenters to ink in presentation mode as well. SMART is another commercial system that supports inking and projected material using an interactive whiteboard \cite{smarttech2017}. While all of these systems combine slides with inking, the underlying slide content remains inherently separate from the ink on top. In contrast, in our system, presenters use ink to reveal underlying slide elements in a flexible, fine-grained way at presentation time.
\val{This sentence would also benefit from clarification of inking: as a gesture vs. output.} 
\wil{Is that better?}

\textbf{Beautifying Ink.} To further assist freeform digital inking, researchers have experimented with different methods to beautify the user's ink strokes. Beautification is applied to meet the requirements of specific scenarios, such as geometric diagrams \cite{igarashi1998pegasus, hse2005recognition, fivser2015shipshape}, hand-drawn pictures~\cite{xie2014portraitsketch, limpaecher2013}, handwriting \cite{zitnick2013handwriting} or mathematical diagrams~\cite{laviola2007mathpad}. Our work does not modify the user's ink stroke per se, but we achieve a similar effect by making the ink stroke disappear gradually and revealing the underlying pre-authored slide content instead.
